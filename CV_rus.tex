%-------------------------
% Resume in Latex
% Author : Jake Gutierrez
% Based off of: https://github.com/sb2nov/resume
% License : MIT
%------------------------

\documentclass[letterpaper,11pt,english,russian]{article}

\usepackage{latexsym}
\usepackage[empty]{fullpage}
\usepackage{titlesec}
\usepackage{marvosym}
\usepackage[usenames,dvipsnames]{color}
\usepackage{verbatim}
\usepackage{enumitem}
\usepackage[hidelinks]{hyperref}
\usepackage{fancyhdr}
\usepackage[english]{babel}
\usepackage[russian]{babel}
\usepackage{tabularx}
\input{glyphtounicode}


%----------FONT OPTIONS----------
% sans-serif
% \usepackage[sfdefault]{FiraSans}
% \usepackage[sfdefault]{roboto}
% \usepackage[sfdefault]{noto-sans}
% \usepackage[default]{sourcesanspro}

% serif
% \usepackage{CormorantGaramond}
% \usepackage{charter}


\pagestyle{fancy}
\fancyhf{} % clear all header and footer fields
\fancyfoot{}
\renewcommand{\headrulewidth}{0pt}
\renewcommand{\footrulewidth}{0pt}

% Adjust margins
\addtolength{\oddsidemargin}{-0.5in}
\addtolength{\evensidemargin}{-0.5in}
\addtolength{\textwidth}{1in}
\addtolength{\topmargin}{-.5in}
\addtolength{\textheight}{1.0in}

\urlstyle{same}

\raggedbottom
\raggedright
\setlength{\tabcolsep}{0in}

% Sections formatting
\titleformat{\section}{
  \vspace{-4pt}\scshape\raggedright\large
}{}{0em}{}[\color{black}\titlerule \vspace{-5pt}]

% Ensure that generate pdf is machine readable/ATS parsable
\pdfgentounicode=1

%-------------------------
% Custom commands
\newcommand{\resumeItem}[1]{
  \item\small{
    {#1 \vspace{-2pt}}
  }
}

\newcommand{\resumeSubheading}[4]{
  \vspace{-2pt}\item
    \begin{tabularx}{0.95\textwidth}[t]{X r}
      \textbf{#1} & #2 \\
      \textit{\small #3} & \textit{\small #4} \\
    \end{tabularx}\vspace{-7pt}
}

\newcommand{\resumeSubSubheading}[2]{
    \item
    \begin{tabular*}{0.9\textwidth}{l@{\extracolsep{\fill}}r}
      \textit{\small#1} & \textit{\small #2} \\
    \end{tabular*}\vspace{-7pt}
}

\newcommand{\resumeProjectHeading}[2]{
    \item
    \begin{tabular*}{0.9\textwidth}{l@{\extracolsep{\fill}}r}
      \small#1 & #2 \\
    \end{tabular*}\vspace{-7pt}
}

\newcommand{\resumeSubItem}[1]{\resumeItem{#1}\vspace{-4pt}}

\renewcommand\labelitemii{$\vcenter{\hbox{\tiny$\bullet$}}$}

\newcommand{\resumeSubHeadingListStart}{\begin{itemize}[leftmargin=0.15in, label={}]}
\newcommand{\resumeSubHeadingListEnd}{\end{itemize}}
\newcommand{\resumeItemListStart}{\begin{itemize}}
\newcommand{\resumeItemListEnd}{\end{itemize}\vspace{-5pt}}

%-------------------------------------------
%%%%%%  RESUME STARTS HERE  %%%%%%%%%%%%%%%%%%%%%%%%%%%%


\begin{document}

%----------HEADING----------
% \begin{tabular*}{\textwidth}{l@{\extracolsep{\fill}}r}
%   \textbf{\href{http://sourabhbajaj.com/}{\Large Sourabh Bajaj}} & Email : \href{mailto:sourabh@sourabhbajaj.com}{sourabh@sourabhbajaj.com}\\
%   \href{http://sourabhbajaj.com/}{http://www.sourabhbajaj.com} & Mobile : +1-123-456-7890 \\
% \end{tabular*}

\begin{center}
    \textbf{\Huge \scshape Наталья Крутых} \\ \vspace{5pt}
    \small +7 910 451 96 76 $|$ \href{mailto:nataliakrutyh@gmail.com}{\underline{nataliakrutyh@gmail.com}} $|$ 
    \href{https://linkedin.com/in/natalia-krutykh}{\underline{linkedin.com/in/natalia-krutykh}} $|$
    \href{https://github.com/awariel7}{\underline{github.com/awariel7}}
\end{center}



%-----------EXPERIENCE-----------
\section{Опыт работы}
  \resumeSubHeadingListStart

    \resumeSubheading
      {Архитектор программного обеспечения}{Сентябрь 2022 -- Март 2024}
      {ООО "Интеллектуальные корпоративные решения"}{ics-it.ru}
      \resumeItemListStart
        \resumeItem{Создан стандартный BI-проект, состоящий из набора бизнес-модулей и платформ, что позволило сократить время запуска новых проектов с 6 месяцев до 3 дней.}
        \resumeItem{На ключевые проекты компании внедрены типовые модули, что улучшило пользовательский опыт и снизило объем поддержки на 30\%.}
      \resumeItemListEnd
      
% -----------Multiple Positions Heading-----------
%    \resumeSubSubheading
%     {Software Engineer I}{Oct 2014 - Sep 2016}
%     \resumeItemListStart
%        \resumeItem{Apache Beam}
%          {Apache Beam is a unified model for defining both batch and streaming data-parallel processing pipelines}
%     \resumeItemListEnd
%    \resumeSubHeadingListEnd
%-------------------------------------------

    \resumeSubheading
      {Руководитель команды разработки}{Декабрь 2021 -- Август 2022}
      {ООО "Интеллектуальные корпоративные решения"}{ics-it.ru}
      \resumeItemListStart
        \resumeItem{Создала первую в компании программу онбординга новых сотрудников в команде интеграции. Результат: 8 из 10 новых кандидатов остались после испытательного срока и проработали более 2 лет.}
        \resumeItem{Предложены и внедрены метрики для определения успешности кандидата по окончании испытательного срока. Результат: были отобраны лучшие кандидаты, которые усилили команду.}
        \resumeItem{Организовано 3 внутренних Tech-Talks и 1 конференция внутри компании для обмена знаниями между командами и отделами. Результат: команда интеграции начала шире использовать возможности инструментов и эффективнее решать задачи.}
    \resumeItemListEnd

    \resumeSubheading
      {Ведущий консультант-разработчик БД}{Август 2015 -- Ноябрь 2021}
      {ООО "Интеллектуальные корпоративные решения"}{ics-it.ru}
      \resumeItemListStart
        \resumeItem{Выросла с позиции младшего разработчика до ведущего.}
        \resumeItem{Разработан алгоритм распределения продаж (Сплит). С момента запуска в продуктивный контур в 2016 году, работает до сих пор, сделано несколько версий в зависимости от типа данных для использования}
        \resumeItem{Разработана панель управления доступом к данным в OLAP-кубе. Реализован инструмент, позволяющий администраторам через UI выдавать доступ сотрудникам к определёнными типам данных в OLAP-кубе, расчёт доступа выполнен с помощью SQL и MDX}
        \resumeItem{Доработан инструмент стандартизации адресов. Реализован контроль денежного баланса за распознавание и дашборд со статистикой, который позволил находить проекты с превышением затрат и отключать им токены без влияния на остальных пользователей}
        \resumeItem{Полностью разработан портал отчётности Tableau для одного из клиентов. В 2018 портал занял первое место среди других регионов компании.}
      \resumeItemListEnd

    \resumeSubheading
      {Лаборант-исследователь}{Ноябрь 2014 -- Май 2015}
      {ФГБУ НИФИ}{nifi.ru}
      \resumeItemListStart
        \resumeItem{Дипломное проектирование. Проанализирована модель жизненного цикла инновационных проектов, создано ПО, позволяющее рассчитать вероятность перехода инновационного проекта на следующую стадию (целесообразность финансирования проекта).Проект написан на языке R с использованием библиотеки Shiny. Использованный математический метод - логистическая регрессия.}
      \resumeItemListEnd

    \resumeSubheading
      {Ведущий инженер-программист}{Январь 2015 -- Апрель 2015}
      {ФГБУ ЦСМС}{cfmc.ru}
      \resumeItemListStart
        \resumeItem{Доработка внутреннего ПО для визуализации данных по позициям рыболовных судов, макросов (MySQL, Delphi, VB)}
        \resumeItem{Работа с картографическим интерфейсом TRANSAS (подготовка информации о местоположении учебных судов для сайта учреждения, подготовка информации для селекторных совещаний Росрыболовства)}
      \resumeItemListEnd
    \resumeSubheading
      {Ведущий специалист}{Сентябрь 2012 -- Сентябрь 2013}
      {Связной Банк}{www.svyaznoybank.ru}
      \resumeItemListStart
        \resumeItem{Разработка (SQL (Oracle), PL/SQL, PL+) и настройка форм отчетности ЦФТ (ежедневной аналитической отчетности банка, МСФО, отчеты по заявкам пользователей)}
      \resumeItemListEnd
    \resumeSubheading
      {Менеджер}{Сентябрь 2011 -- Январь 2012}
      {МГОУЭСИ}{mesi.ru}
      \resumeItemListStart
        \resumeItem{Ведение личных дел студентов, ведение базы данных обучающихся студентов, выдача информации из личного дела студента по запросу внутренних структур университета, оформление дел и сдача в архив}
      \resumeItemListEnd
  \resumeSubHeadingListEnd

%-----------EDUCATION-----------
\section{Образование}
  \resumeSubHeadingListStart
    \resumeSubheading
      {Технический Университет Мюнхена}{Мюнхен, Германия}
      {Магистратура по направлению "Инженерия и аналитика данных" (неоконченное)}{Апр 2024 -- Сен 2024}
    \resumeSubheading
      {Московский Государственный Университет Экономики, Статистики и Информатики (МЭСИ)}{Москва, Россия}
      {Математическое обеспечение и администрирование информационных систем}{Сен 2010 -- Июл 2015}
  \resumeSubHeadingListEnd


%
%-----------PROGRAMMING SKILLS-----------
\section{Технические навыки}
 \begin{itemize}[leftmargin=0.15in, label={}]
    \small{\item{
     \textbf{Языки}{: Python, SQL, R, LaTex} \\
     \textbf{Фреймворки}{: Flask, Django} \\
     \textbf{СУБД}{: SQL Server, PostgreSQL, Clickhouse, Redis} \\
     \textbf{Архитектура}{: REST, UML, C4}\\
     \textbf{Распределённые системы}{: Apache Kafka} \\
     \textbf{Инструменты разработчика}{: Git, Docker, VS Code, PyCharm, Liquibase, Grafana}
    }}
 \end{itemize}

%
%-----------Other SKILLS-----------
\section{Знание языков}
 \begin{itemize}[leftmargin=0.15in, label={}]
    \small{\item{
     \textbf{Русский}{: Родной} \\
     \textbf{Английский}{: C1 (IELTS Academic 7.5)} \\
     \textbf{Немецкий}{: A2 (Goethe-Zertifikat)}
    }}
 \end{itemize}

 %
%-----------Other SKILLS-----------
\section{Повышение квалификации}
 \begin{itemize}[leftmargin=0.15in, label={}]
    \small{\item{
     \textbf{Mentor in Tech 6.0}{: Woman in Tech Russia\\ Окт 2024 - Фев 2025} \\
     \textbf{OAIS: Основы архитектуры и интеграции информационных систем}{: УЦ "Коммерсант"\\ 2023} \\
     \textbf{Математика для анализа данных}{: ВШЭ\\ 2019} \\
     \textbf{Python для анализа данных}{: ВШЭ\\ 2019}
    }}
 \end{itemize}


%-------------------------------------------
\end{document}